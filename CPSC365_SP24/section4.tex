\documentclass[11pt]{article}
\usepackage[margin=1.5in]{geometry}
\usepackage{fontspec} % Advanced font selection
\setmonofont[Scale=MatchLowercase]{Inconsolatazi4}
\usepackage{amsfonts, amsmath, amssymb, amsthm} % AMS Math Packages
\usepackage{mathtools} % Mathematical tools for amsmath
\usepackage{array, booktabs} % Extends the array and tabular environments
\usepackage{environ} % Interface for environments
\usepackage{parskip, setspace} % Paragraph indentation and line spacing
\usepackage{color} % Color control
\usepackage{colortbl} % Color to tables
\usepackage[x11names]{xcolor} % Color extensions 
\usepackage{soul} % Letter spacing, underlining, striking out 
\usepackage{hyperref} % Support for hypertext
\hypersetup{pdfborder=0 0 0, citebordercolor=Gray0, urlbordercolor=Gray0, linkbordercolor=Gray0}
\usepackage[capitalise]{cleveref} % Better cross-referencing
\usepackage{mdframed} % Framed environments that can split
\usepackage{fancybox, fancyhdr} % Variations of fbox and header/footer controls
\usepackage{tcolorbox} % Colored boxes 
\tcbuselibrary{breakable}
\usepackage{tabulary} % Tabular environments with variable width columns balanced
\usepackage{scrextend} % Extends KOMA-Script classes to standard classes
\usepackage{graphicx} % Enhanced support for graphics 
\usepackage[shortlabels]{enumitem} % Control layout of itemize, enumerate, description
\usepackage{tikz} % Graphics creation
\usetikzlibrary{shapes.geometric} 
\usepackage{fontspec, svg} % Include and extract SVG pictures
\usepackage[linesnumbered,lined,boxed,commentsnumbered]{algorithm2e} % Algorithms creation
\usepackage{listings} % Typeset source code listings
\usepackage{dcolumn} % Alignment on decimal points
\usepackage{float} % Improvements to floating objects
\usepackage[font=scriptsize,labelfont=bf]{caption}
\usepackage{pdflscape}
\usepackage[export]{adjustbox}

\allowdisplaybreaks

\renewcommand{\baselinestretch}{1.25}
\sethlcolor{Cyan3!50}

\DeclarePairedDelimiter\floor{\lfloor}{\rfloor}
\DeclarePairedDelimiter\ceiling{\lceil}{\rceil}

\newcolumntype{C}[1]{>{\centering\let\newline\\\arraybackslash\hspace{0pt}}m{#1}}

\title{Section 4: Dijkstra's Algorithm}
\date{February 16, 2024}
\author{Amy Zhao}

\begin{document}
\renewcommand\labelitemi{$\vcenter{\hbox{\tiny$\bullet$}}$}
\renewcommand\labelitemii{$\vcenter{\hbox{\tiny$\bullet$}}$}

\counterwithin{figure}{section}

\maketitle

\section{Dijkstra's Algorithm}
At a high level, Dijkstra's algorithm is an algorithm which finds the shortest path between a starting vertex and another vertex in the graph. During each iteration, we expand the search by identifying the closest not-yet-processed vertex from the starting vertex. 

\IncMargin{2em}
\begin{algorithm}[H]
    \footnotesize
    \DontPrintSemicolon
    \setstretch{1.125}
    \SetKwProg{Fn}{Function}{:}{}
    \SetKwFunction{Dijkstra}{Dijkstra}
    \SetKwInOut{Input}{Input}\SetKwInOut{Post}{Post}
    \Input{Graph $G = (V, E)$ given as an adjacency list, starting vertex $s$}
    \Post{Shortest distances from $s$ to $v \in V$, stored in dist[$v$]}
    \BlankLine
    \Fn{\Dijkstra{$G = (V, E)$, $s$}}{
        $Q \gets$ new priority queue\; 
        \BlankLine \BlankLine 
        \For{$v \in V$}{
            distance[$v$] $\gets \infty$ \;
            previous[$v$] $\gets$ null 

            \uIf{$v \neq s$}{
                add $v$ with its distance to $Q$\; 
            }
        }

        \BlankLine \BlankLine
        distance[$s$] $\gets$ 0 \; 

        \BlankLine \BlankLine
        \While{\upshape $Q$ is not empty}{
            $u \gets$ vertex in $Q$ with minimum distance \;
            \BlankLine \BlankLine
            \For{\upshape each unvisited neighbor $v$ of $u$}{
                temp $\gets$ distance[$u$] $+$ weight[$u$, $v$]\;
                \uIf{\upshape temp $<$ distance[$v$]}{
                    distance[$v$] $\gets$ temp\; 
                    previous[$v$] $\gets$ $u$\;
                }
            }
        }
    }
\end{algorithm}\DecMargin{2em} 


\section{What Can Go Wrong?}
\subsection{Negative Edges}
In this section, we illustrate a few examples when performing Dijkstra's will yield the wrong answer. First, suppose we have a graph $G = (V, E)$ with edge weights $L = \{\ell_e : e \in E\}$, and we allow for the possibility of negative edges. Assume there are no negative cycles. Evidently, we know that Dijkstra's algorithm doesn't work on graphs with negative edges. Consider a modification to a graph with at least one negative edge:
\begin{itemize}
    \item Find the minimum edge weight in $L$, denoted by $\ell^* < 0$.
    \item For all edges, change the edge weight to $\ell_e' = \ell_e + |\ell^*| + 1$, so that all of the new edge weights are positive. Denote the new set of edges as $L' = \{\ell'_e: e \in E\}$.
    \item Run Dijkstra's algorithm as usual with the new set of edge weights. 
\end{itemize}
Will this change yield the shortest paths from a source vertex to all other vertices? Can we find a simple example or counterexample first?

\begin{figure}[H]
    \centering 
    \includesvg[width=0.5\textwidth, pretex=\scriptsize]{figures/incorrect_dijkstra.svg}
    \caption{Example of a single graph where Dijkstra's algorithm will not yield the correct answer on the modified graph. The shortest path in the original graph from $a$ to $c$ is $a \to b \to c$, while the shortest path in the modified graph is $a \to c$.}
\end{figure} 

We claim that the proposed modification does not find the shortest path and provide a proof. For notation purposes, let $u$ be the source vertex, and let $v \in V$ be any other vertex in the graph. Let $W(P)$ be the sum of edge weights on a path $P$ using the edge weights from the original graph, and let $W'(P)$ be the sum of the edge weights on a path $P$ using the modified weights:
\begin{align*}
    W(P) &= \sum_{e \in P} \ell_e & W'(P) &= \sum_{e \in P} \ell'_e.
\end{align*}
We set $P$ to be the shortest path using the original edge weights. First, consider a path $P$ containing $k$ edges and the resulting sum on the modified edges:
\begin{align*}
    W'(P) = \sum_{e \in P} \ell_e' &= \sum_{e \in P} (\ell_e + |\ell^*| + 1) \\ 
    &= \left(\sum_{e \in P} \ell_e \right) + k(|\ell^*| + 1) \\ 
    &= W(P) + k(|\ell^*| + 1).
\end{align*} 
Consider a second path $Q$ from $u$ to $v$ with $j < k$ edges but greater weight $W(Q) > W(P)$. We write the sum $W'(Q)$ as:
\begin{equation*}
    W'(Q) = \sum_{e \in Q} \ell_e' = W(Q) + j(|\ell^*| + 1). 
\end{equation*} 
These relationships yield the following implication:
\begin{align*}
    W(Q) - W(P) &< (k-j) (|\ell^*| + 1) \\ &\Longrightarrow W'(Q) = W(Q) + j(|\ell^*| + 1) < W(P) + k(|\ell^*| + 1) = W'(P).
\end{align*}

In general, $W(P) < W(Q)$ does not imply $W'(P) \leq W'(Q)$, and thus the modification does not always return the shortest path with respect to the original edge weights. 

\subsection{Adding a Constant}
Now, consider a positive weighted graph, and assume we use Dijkstra's algorithm to compute a shortest path $P$ between two vertices $u$ and $v$. Will Dijkstra's algorithm yield the same path if we add a positive constant to every edge?

Suppose the original graph has two possible paths from $s$ to $t$: one which includes 50 edges with $\ell_e = 1$ and one which includes a single edge with $\ell_e = 51$. When we run Dijkstra's algorithm on this graph, the shortest path will be the first path because it has a total weight of 50, as compared to the second which has total weight of 51. 

If we add a constant to each edge of the graph, say $c = 50$, the total weight of the first path will be $50(1 + 50) = 2550$, while the weight of the second path will be $51 + 50 = 101$. Thus, this clearly illustrates why Dijkstra's will not work on this modified graph to return the shortest path length of the original graph. Paths with a greater number of edges will have a greater amount of weight added:
\begin{align*}
    W'(P) &= \sum_{e \in P} (c + \ell_e) = |P|(c) + \sum_{e \in P} \ell_e = |P|(c) + W(P), \\
    W'(Q) &= \sum_{e \in Q} (c + \ell_e) = |Q|(c) + \sum_{e \in Q} \ell_e = |Q|(c) + W(Q). 
\end{align*}
$|P|(c) \gg |Q|(c)$ if $|P| \gg |Q|$, and this is solely dependent on the number of edges in the path. 

\subsection{Scaling by a Constant}
Taking the same setup as in the previous section, we decide to scale each edge by a constant positive factor, instead of adding a constant. Will Dijkstra's algorithm work on this modified graph to produce the shortest path from the original?

Let $P$ be the shortest path from $s$ to $t$ in the original graph, and let $Q$ be an alternate path from $s$ to $t$ in the original graph. Assume also that the number of edges in the two paths may be different. We denote the weight of the original path with $W$ and the weight of the modified path with $W'$. 

As $P$ is the shortest path, we have $W(P) < W(Q)$. We'll look at the scaling modification:
\begin{align*}
    W'(P) &= \sum_{e \in P} (c \cdot \ell_e) = c \cdot \sum_{e \in P} \ell_e = c \cdot W(P). \\
    W'(Q) &= \sum_{e \in Q} (c \cdot \ell_e) = c \cdot \sum_{e \in Q} \ell_e = c \cdot W(Q). 
\end{align*}
Thus, we have:
\begin{equation*}
    W'(P) = c \cdot W(P) < c \cdot W(Q) = W'(Q) \Longrightarrow W'(P) < W'(Q),
\end{equation*}
which implies that we will obtain the same shortest path in the modified graph when we run Dijkstra's. 

\section{Example: Fuel Capacity}
Suppose we have a set of cities connected by highways, given in the form of an undirected graph $G = (V, E)$. Each highway $e \in E$ has a specified length $\ell_e > 0$. You are planning a road trip and want to determine if you can travel from city $s$ to city $t$ with a car that has a fuel capacity of $L$ miles. You can refuel at each city, but not between, which means you can take a route if every one of its highways is smaller in length than $L$. 

\subsection{Reachability}
If we know the car's fuel capacity, how do we determine if we can reach $t$ from $s$? Revisiting the previous discussion section's examples, we can consider modifying the graph by taking out any edges that are longer than $L$. Running DFS on the resulting graph will check reachability between $s$ and $t$, yielding the correct answer. Modification and DFS are $O(|V| + |E|)$ runtime.

\subsection{Determining Needed Capacity}
However, we face a trickier problem if we don't know the car's capacity. Suppose we are purchasing a new vehicle and want to ensure we buy a car with enough fuel capacity to travel from $s$ to $t$. Notably, we can refuel at each city as needed, so the car's capacity need only to be at least as large as the longest edge of the smallest path from $s$ to $t$. In other words, the fuel capacity needed is constrained by the $s$ to $t$ path whose longest edge is the smallest; we'll call this the smallest max length path. 

To determine this length, we adapt Dijkstra's algorithm. At each vertex $u$, we track the constraining edge, instead of the distance, in $m$[$u$]. Every time we dequeue a vertex $u$, we compare the length of the edge $e = \{u, v \}$ with the current smallest max length:
\begin{itemize}
    \item $\max\{\ell_2, m[u]\} < m[v]$: set $m[v] = \max\{\ell_e, m[u]\}$ and rebalance the priority queue; do nothing otherwise.
\end{itemize}

The algorithm returns the value at $m[t]$, which will accurately report the constraining edge in the smallest path.



\IncMargin{2em}
\begin{algorithm}[H]
    \footnotesize
    \DontPrintSemicolon
    \setstretch{1.125}
    \SetKwProg{Fn}{Function}{:}{}
    \SetKwFunction{FuelConstraint}{FuelConstraint}
    \SetKwFunction{ExtractMin}{ExtractMin}
    \SetKwInOut{Input}{Input}\SetKwInOut{Output}{Output}
    \Input{Graph $G = (V, E)$ given as an adjacency list with weights $\ell_e > 0$ for $e \ in E$, starting vertex $s$, target vertex $t$}
    \Output{Smallest max length path}
    \BlankLine
    \Fn{\FuelConstraint{$G = (V, E)$, $s$, $t$}}{
        \For{$v \in V$}{
           \uIf{$v = s$} {
                $m[v] \gets 0$ \;
           } 
           \uElse{
                $m[v] \gets \infty$ \; 
           }
        }

        \BlankLine \BlankLine
        $Q$ $\gets$ new priority queue with objects $v \in V$, associated with priority $m[v]$ \; 

        \BlankLine \BlankLine 
        \While{\upshape $Q$ is not empty}{
            $v \gets$ \ExtractMin{$P$} \; 

            \For{\upshape $w \in \text{Adj}[v]$}{
                \uIf{$m[w] > \max \{\ell_{vw}, m[v] \}$}{
                    $m[w] \gets \max \{\ell_{vw}, m[v] \}$ \; 
                    Adjust priority key for $w$\; 
                }
            }
        }
        \BlankLine \BlankLine 
        \Return{$m[t]$}
    }
\end{algorithm}\DecMargin{2em} 

Tracking the smallest max length path does not alter the runtime of Dijkstra's algorithm, so the overall time complexity remains $O((|E| + |V|) \log(|V|))$.

\section{Example: Number of Shortest Paths}
We know that Dijkstra's algorithm is used to find the shortest paths in a weighted undirected graph. Is it possible to augment the algorithm to count the number of shortest paths from $s$ to $t$? Recall that in running Dijkstra's algorithm, all other vertices with a shorter path from $s$ will have already been explored before we expand vertex $a$ in Dijkstra's algorithm. This means that we know the shortest path from $s$ to these vertices is known at this point, and we can simply record additional information about the shortest paths based off a couple comparisons.

At some iteration of Dijkstra's algorithm, let's consider the expansion of vertex $u$ on the edge $\{u, v\}$ with length $\ell_e$:
\begin{itemize}
    \item $\text{dist}[v] > \text{dist}[u] + \ell_e$: We find a shorter path to $v$, so we set $\text{dist}[v] = \text{dist}[u] + \ell_e$, and we update the number of paths to $v$ to be equal to the number of paths from $s$ to $u$, $\text{numpaths}[v] = \text{numpaths}[u]$.
    \item $\text{dist}[v] = \text{dist}[u] + \ell_e$: We find a path with the same distance as previously recorded, so there are no updates to make to the distance array. However, we increase the number of paths to $v$, $\text{numpaths}[v] += \text{numpaths}[u]$, as we care about the path distances, rather than the actual edges of the paths.  
\end{itemize}
The algorithm returns the value at $\text{numpaths}[t]$, which will accurately report the number of shortest paths because the value is updated in parallel with any updates to the distance array. This means that whenever $v$ has the smallest distance value, there are no other unexplored paths from $s \to v$ that could decrease the value we've stored at dist[$v$].

\IncMargin{2em}
\begin{algorithm}[H]
    \footnotesize
    \DontPrintSemicolon
    \setstretch{1.125}
    \SetKwProg{Fn}{Function}{:}{}
    \SetKwFunction{NumShortestPaths}{NumShortestPaths}
    \SetKwFunction{ChangeKey}{ChangeKey}
    \SetKwInOut{Input}{Input}\SetKwInOut{Output}{Output}
    \Input{Graph $G = (V, E)$ given as an adjacency list, starting vertex $s$, target vertex $t$}
    \Output{Number of shortest paths from $s$ to $t$}
    \BlankLine
    \Fn{\NumShortestPaths{$G = (V, E)$, $s$, $t$}}{

        \For{$v \in V$}{
            dist[$v$] $\gets$ $\infty$ \; 
            numpaths[$v$] $\gets$ $0$ \; 
        }

        \BlankLine \BlankLine 
        dist[$s$] $\gets$ $0$ \; 
        numpaths[$s$] $\gets$ $1$\;
        $Q$ \gets new priority queue

        \BlankLine\BlankLine
        \While{\upshape $Q$ is not empty}{
            $v \gets$ vertex in $Q$ with minimum distance\; 

            \BlankLine \BlankLine
            \For{\upshape $w \in \text{Adj}[v]$}{
                \uIf{\upshape $\text{dist}[w] > \text{dist}[v] + \ell_{vw}$}{
                    $\text{dist}[w] \gets \text{dist}[v] + \ell_{vw}$\; 
                    \ChangeKey{$P$, $w$}\; 
                    $\text{numpaths}[w] \gets \text{numpaths}[v]$\; 
                }
                \uElse{
                    $\text{numpaths}[w] \gets \text{numpaths}[w] + \text{numpaths}[v]$\; 
                }
            }
        }
        \Return{\upshape numpaths[$t$]}
    
    }
\end{algorithm}\DecMargin{2em} 


The runtime of the algorithm is the same as Dijkstra's, $O((|E| + |V|)\log|V|)$, as the comparisons and updates to the numpaths array don't impose any additional dominating runtime. 



\end{document}